\section{Conclusion}\label{sec:conclusion}

This section concludes the report, going over in summary what it contains and what we learned from the project. We also speak about our experience with the project and how working with people from all over the world affected the project. 

In section \ref{sec:background}, we explained why \emph{Project Discovery} is a Game with a purpose, and how we will use it to get ordinary citizens to classify all of the images of human cells in the subcellular atlas. We also discuss the partners of the project, the Swedish research program the Human Protein Atlas (HPA), the Swiss startup Massively Multiplayer Online Science (MMOS), and the game developer and producer of \emph{EVE Online}, CCP Games.

In section \ref{sec:projectdiscovery}, we explained the gameplay of \emph{Project Discovery}, describing in detail the game's prominent features, especially the tutorial phase and the difference between the known results screens and the unknown result screens. Next we discussed the network communications necessary for Project Discovery to connect to the API that MMOS provided, and the architecture of the \emph{Project Discovery} code within the \emph{EVE Online} client. Finally we went over the \emph{Project Discovery} website we designed alongside the client, to provide information about \emph{Project Discovery} and the science behind the project.

In section \ref{sec:userevaluation}, we went into detail about the preliminary, in-house user test that we performed within CCP Games, with their employees as participants. We explained the implementation of the test and its results.

In section \ref{sec:workscheduleandflow}, we explained the work flow of the project, how we scheduled the time we had available for the project, and how we actually wound up spending the time we had. We also explained the methodology we used for managing the work on this project as well as managing the team itself, and we discussed how it worked out for us.

Finally, in section \ref{sec:futurework}, we discussed the future of \emph{Project Discovery}, and what it entails, whether or not the project will be continued by CCP Games, and how the \emph{Project Discovery} client has actually been carefully designed with the express goal in mind of solving other scientific problems as well, not just the subcellular atlas, a small part of The Human Protein Atlas.

The status of \emph{Project Discovery} as of December 2015 is really positive. The game has a strong support from CCP and has already been released on the Singularity test server. Hjalti will continue working on the game as an intern at CCP, with the goal of releasing it on the Tranquility server cluster. Furthermore, the game has received considerable attention both in Iceland and internationally. The game was mentioned in \emph{Nature Magazine}~\cite{nature} which is one of the most established and respected scientific journals in the world. The game featured heavily at the EVE Vegas conference in October 2015, where it was presented at the keynote speech and it was also playable for attendees of the conference. After EVE Vegas there were several articles about the game online, Discovery.com~\cite{discovery} being the most prominent. The project also garnered attention from the university, being featured in both the RU Magazine~\cite{rumag} and RU.is~\cite{ruis}, it was also represented at the CADIA (Reykjavík University's artificial intelligence research center) festival in October 2015. 

This project was nothing short of an amazing experience, getting to go outside of the university and working in the field of computer science with such incredible partners can hardly be described. We felt lucky to be a part of such a big project, that would surely touch hundreds, if not thousands of people if we were successful in the development of \emph{Project Discovery}, which surely added a level of stress. It has been very rewarding and motivating to work on a project like this that has garnered the type of attention that it has, being presented in the keynote at EVE Vegas 2015, and being mentioned in Nature magazine are such huge things for students like us to be part of. All this aside, we met our intended goals and are happy with the outcome of the project, we would not change much if we could do it again, although being preoccupied with other university courses was hindering in some cases.

Finally, we would like to thank all our partners, CCP Games for hosting this project and supporting us all the way, MMOS for working with us this closely to achieve the best quality of the project, and the Human Protein Atlas for providing us with material to best teach new players how to classify the images. Special thanks go to Pétur Örn Þórarinsson from CCP, our project manager for helping us with game design and prioritizing tasks, David Thue from CADIA for connecting us with the project and supporting it, Attila Szantner from MMOS for the weekly meetings and coming to Iceland to work with us more closely, Emma Lundberg from the HPA for coming to Iceland and helping us closely with how to best teach classification, Team Space Glitter for answering all our questions, and assisting us with \emph{EVE} related programming problems, and finally, Sergey Trubetskoy for designing and providing all the assets for \emph{Project Discovery}.

Everyone was a delight to work with and we could not have asked for a better team for this project. Surely, the experience we gathered from this project will pay off immensely in the future. This project was very special to us and we hope it will become something great within the \emph{EVE Online} world, helping the scientific community and setting a precedent for other game developers to follow, designing games with the purpose of helping the world.