\section{Future Work}\label{sec:futurework}
Development of \emph{Project Discovery} will not end with this final project. Hjalti has already accepted an internship at CCP and will continue to develop the game with the ultimate goal of releasing it on Tranquility, \emph{EVE Online}'s main server cluster \cite{tranquility}. Before that can be realized, a few objectives must be met. The game is currently out on Singularity \cite{singularity}, \emph{EVE Online}'s biggest test server, which allows thousands of players to access the game. When those players play the game it will generate a huge amount of data and feedback, which could be very influential on the future and direction of the game. Responding to that data and feedback is one of the objectives that must be met before the Tranquility release. The other objectives are features that were planned at the start of this final project, but because of time constraints and changes to the focus of the game, were not implemented before the Singularity release. Those features include leaderboards, daily challenges and a more interactive tutorial. 

The subcellular protein atlas, the current project from the Human Protein Atlas that \emph{Project Discovery} is being utilized for, has a finite number of images. Therefore analysis of those images will at some point conclude, which could actually happen fairly early in the lifespan of the game if \emph{EVE Online} players embrace the feature. Once that point is reached there is the possibility to adapt \emph{Project Discovery} to work with other citizen science projects. The Human Protein Atlas, for an example, has other atlases that could potentially be the focus of \emph{Project Discovery} in the future. MMOS designed their API with this in mind, they expect the subcellular atlas to be the first of many scientific projects that \emph{Project Discovery} assists. The game client has also been structured with regard to this possibility. Any project that involves analysis of images in a similar way to the subcellular atlas could therefore replace that project once it has been completed. To use the client for another project would certainly need substantial customization for that project, so it's not something that is quick and easy to do, but it is definitely possible.