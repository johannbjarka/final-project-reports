\section{Work schedule and flow}\label{sec:workscheduleandflow}

In this section we deviate from talking about citizen science and Project Discovery and go over the work schedule and flow we followed during the project. We will discuss the work schedule we put together and how we managed to follow that schedule. We will also go into the methodology we decided to use to aid us during the project and how that influenced the project. As a part of that we will show the burndown chart for the project as a whole and discuss all of the sprints that we finished while working on the project.

% Erum að breyta um gír og fjalla um flæðið í verkefninu 

\subsection{Initial plan}
	\subsubsection{Initial schedule}
		When we started the project, we estimated how much time we could afford to spend on the project, taking into account the workload from other courses. We ended up estimating that we had a total of 654 hours to spend during the entire project. A breakdown of how we scheduled those hours can be seen in table \ref{table:hours}. Work on the project was essentially split into two phases, the first one being the 12 week semester, from the 17th of August to November the 11th, in which other courses affected work on the project. The second one being the three week semester, from the 26th of November to December the 16th, where we could focus all our attention on the project. 
	  
	\begin{table}[H]
	  	\centering
		\addvbuffer[12pt 12pt]{\begin{tabular} {| l |  c | c |}
		\hline & Hjalti &  Jóhann\\
		\hline Total for 12 week semester & 162 hours & 192 hours \\
		\hline Total for 3 week semester & 150 hours & 150 hours \\
		\hline Total for the project & 312 hours & 342 hours \\
		\hline Total for the project combined & \multicolumn{2}{c |}{654 hours} \\ 
		\hline
		\end{tabular}}
		\caption{Breakdown of the initial schedule.}
		\label{table:hours}
	\end{table}

	\subsubsection{Initial stories}


\subsection{Methodology}
	For this project we utilized the Scrum methodology to help us document and organize the project and its progress. The appointed Scrum Master was Jóhann and the Product Owner was Hjalti. Pétur Örn Þórarinsson, a lead game designer at CCP, served as our Project Manager. Since the team only consisted of two people, the Scrum Master and Product Owner made up the the whole team. The project consisted of seven, two week long sprints, with the work divided so that the team could direct enough attention towards other courses, as well as work on the project.

	Daily Scrum meetings were held at CCP headquarters every work day. During the three week term they took place at 11:30 AM, but they were held at various times of the day during the 12 week term, as the team wasn't always present throughout the day. We chose Scrum because it gave us good documentation on the team's progress and it gave us an indication of what we could achieve according to our velocity. With each iteration we could track what was going well and what needed improving. Scrum gave us the platform to review and improve on our process during each iteration.

\subsection{Progress during the project}

	\subsubsection{Actual hours spent}

	TODO: Insert images from toggl of hours spent once work is done.
	
	\subsubsection{Actual stories}

	\subsubsection{Project burndown chart}
		
		\begin{figure}[H]
		  \centering
		  \graphicspath{ {./graphics/} }
		  \centerline{\includegraphics[scale=0.55]{bdchart.png}}
		  \caption{\label{fig:bdchart} The burndown chart for the whole project.}
		\end{figure}

	\subsubsection{Sprint summary}

\subsection{Schedule summary}