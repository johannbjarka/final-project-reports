\section{Project Discovery}\label{sec:projectdiscovery}

	In this section we discuss the game itself. We explain how it communicates with MMOS' API, the code architecture of the game client and how it fits into the EVE Online client. We also go over how the gameplay works and lastly consider the website we built to go along with the game's launch.

\subsection{Network Communications}
	The Project Discovery client utilizes a connection with two different parts, the EVE server, and the MMOS RESTful API, which uses the Amazon CloudFront CDN and an Elastic Beanstalk container.

	When a player opens up the client, it automatically contacts the EVE server, which connects to the MMOS API and asks for a new task for the player. The MMOS API then decides the new task and fetches it from the Elastic Beanstalk container, and sends it to the EVE server, which finally sends it to the EVE client. The task object includes a security pass that the client needs to access the URL's for each image that he needs for the task. Each time the client needs an image for that task, it contacts the EVE server, which contacts the MMOS API with the security pass, if that pass is still active, the Amazon CloudFront CDN returns a signed URL that leads to that image, the image itself is only accessible for a few seconds, as a security feature, so the client needs to cache the image in case the player needs to access it again.

	The MMOS API authentication uses a HMAC-SHA256 method, based on Amazon authentication methods. The server has an API key and secret, which are changed at regular intervals. The API key and secret are used to generate a signing key, this adds an additional layer of security, since every message is hashed with different values.

\subsection{Architecture}

	%Skrifað í stackless python, notast við UI library CCP, nefna klasana sem tilheyra leiknum, server kóðann

\subsection{Client within a client}

\subsection{Gameplay}

	\subsubsection{Tutorial}
		When a player opens up Project Discovery for the first time he must start by going through a tutorial phase. The purpose of the tutorial phase is to allow the player to familiarize himself with the controls and the user interface of the game as well as teaching him the basics of analyzing subcellular images. To achieve this we first introduce the player to the controls by using tooltips which explain the functionality of each control element, one at a time, while also allowing the player to interact with the controls. Once the player has gone through the tooltips he gets his first image to analyze, this image is the first stage of seven in the tutorial. Each sta

\subsection{The website}
	One of the goals of the project was to launch a Project Discovery website along with the release of the game. The website serves as an information hub for the project. Interested parties can get information about the game, the science behind the game and our partners, MMOS and the Human Protein Atlas. The website also displays videos related to the game and screenshots of the game in action. To build the website we used HTML5, Less and JavaScript. We were provided with a template from web developers within CCP which helped us make the website look professional and consistent with other EVE Online websites.

%Setja inn mynd af vefsíðu?