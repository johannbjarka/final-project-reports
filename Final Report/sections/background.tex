\section{Background}\label{sec:background}
	In this section we discuss what a game with a purpose is, and why our project falls within that category.
	Next we outline the previous work already committed to the project by previous students.
	We discuss MMOS (Massively Multiplayer Online Science), CCP Games, The Human Protein Atlas, and what they have done for the project. And finally we show how this project helped solve real world problems in the biological sciences.

\subsection{Game with a purpose}
	In the modern world it is one of the facts of life that people play video games, either competitively or casually, and the hours humans spend each year on video games is in the billions. What if we could channel some of that time and energy into something really useful?

	Many scientific research problems require the analysis of huge amounts of data. Scientists are literally sitting on mountains of data that can not conceivably be processed in their lifetime. One might think we could use computers to solve these problems, but despite colossal advances in computers over the last 50 years, and machine learning recent years, these algorithms often fail to provide good enough quality for scientific research.

	Because humans possess complex conceptual intelligence, perceptual capabilities and pattern recognition that algorithms can not yet mimic; computers can not be used to solve these problems for us. If we could use a large number of ordinary humans as processors in a distributed computer-like system, we could solve real world large scale problems that would otherwise be considered impossible or too time consuming.

	Seeing how popular video games are, what if we designed a game, whose purpose is outsourcing computationally difficult functions to humans in an entertaining way? These \emph{games with a purpose} are ways for regular people outside the scientific community to contribute to scientific research in various fields, and entertaining themselves in the meantime.

	This concept has already been proven useful before in games such as \emph{FoldIt}, where players compete to find the lowest energy state of a protein, thus helping research in the prediction of protein structures, and \emph{Galaxy Zoo}, where players help each other classifying astral bodies in order to find out how galaxies are formed.

	% TODO: Insert references to https://www.cs.cmu.edu/~biglou/ieee-gwap.pdf

\subsection{Previous work}
	The project was a continuation of work done by Reykjavík University graduates Gunnar Þór Stefánsson and Þór Adam Rúnarsson. They worked on it as their final project for the spring semester of 2015, this section outlines what work they did on the project. 

	Gunnar Þór Stefánsson and Þór Adam Rúnarsson continued their work with a grant from Rannís during the summer of 2015. Gunnar Þór Stefánsson left the project in the beginning of July and Hjalti Leifsson replaced him. The work continued until Þór Adam Rúnarsson left the project and Jóhann Bjarkason joined in the end of August, where this final project began.

	When we started the project, the interface had the hexagonal category selection and the header and footer elements. The game already had a few features implemented, such as players being able to receive images, selecting their appropriate categories and submitting their classifications.
	A simple rewarding system was already in place, rewarding players with in game currency (ISK), experience points (XP) and loyalty points (LP) when players submitted a solution. The reward was based on their score for the image they just classified, but since those images were \emph{training images} (i.e. Images that experts have already solved), we already knew the solution and could grade the player on that. However, that needed to be changed later on because when the client got an image that we did not know the solution for, no reward was given.
	Finally, a rudimentary tooltip based tutorial phase was implemented, which only explained the interface to the user, but nothing about how to actually play the game effectively.

\subsection{CCP and EVE Online}
	CCP Games is the company that owns the MMORPG EVE Online, which is home to over 7,000 star systems and hundreds of thousands of players. They hosted us while working on integrating this project into EVE Online and provided a spot in a team called \emph{Team Space Glitter}, where we could ask for programming and design advice, which was very well received.

	EVE Online was perfect for this kind of project, because it's players are used to pretty technical and difficult challenges already in the game. The images we selected also fit the space theme of the game.

\subsection{MMOS}

	Massively Multiplayer Online Science (MMOS) - the name of the concept, and the name of the Swiss company behind it – challenges the way that citizen science is	carried out today. MMOS doesn’t use gamification, but instead cooperates with major	computer game companies to access a resource that was unavailable to citizen science before: the billions of hours each week that people spend playing computer	games. In World of Warcraft alone, players collectively spent almost 6 million years playing by 2010!

	While current approaches to using games for citizen science either create entire new games or ``gamify'' tasks for each specific research project (e.g., Play to Cure), MMOS seeks to leverage the enormous potential that already exists in major computer games. 

	By injecting research projects into existing, popular games as seamless parts of their gaming experiences (including mechanics, narrative, and visuals) we can benefit from the time that players spend on games. Even if we can only direct a small fraction of this time to scientific projects, we can deliver a powerhouse of human computation to scientific research.

	This approach addresses the core challenges of citizen science for the following reasons:

	\begin{itemize}
	  \item {\bf Engagement:} There is a massive amount of user base already engaged, motivated to
	  solve complex tasks.
	  \item {\bf Motivation \& Retention:} As the solution is not a secret to the gamer, the intrinsic motivation of
	  helping science is very important, but to keep up the long-term engagement,
	  integrating it with in-game reward systems and other game mechanics is vital
	  and adds an additional layer of motivation.
	  \item {\bf Separation of Concerns:} Researchers do their research work
	  (defining tasks, analysing results, communicating in research related issues)
	  and professional game designers create the in-game experiences.
	  \item {\bf Reliability:} Popular, massively-multiplayer games tend to last for many years, with some running for more than a decade
	  with hundreds of thousands of devoted players.
	\end{itemize}

	We believe that this approach offers a win-win situation that benefits all parties: scientists, game players, and the gaming industry at large. 

\subsection{Human Protein Atlas}
	The Human Protein Atlas is a project based in Sweden funded by the Knut and Wallenberg foundation in order to allow for a systematic exploration of the human proteome using antibody-based proteomics. %\cite{http://www.proteinatlas.org/about/project}

	This research project is sitting on millions of unclassified images of cells, subcells, cancer tissue and normal tissue. The classification of these images is vital in the understanding of diseases such as cancer, and also in the understanding of the role protein plays in cells.
