\section{Background}\label{sec:background}
	In this section we discuss what a Game with a purpose (GWAP) \cite{gwap} is, why our project falls within that category, we name the affiliates to this project, and what they have done for the project. Finally, we outline the previous work already committed to the project by previous students, and where we picked up the project.

\subsection{Game with a purpose}
	In the modern world it is one of the facts of life that people play video games, either competitively or casually, and the hours humans spend each year on video games is in the billions. If we could channel some of that time and energy into something really useful, we could do some amazing things, like solving large scale, real world scientific problems.

	Many scientific research projects require the analysis and/or classification of huge amounts of data, in these cases, the few scientists on the projects can not conceivably process the data themselves in their lifetime. One might think we could use computers to solve such problems for us automatically, but despite colossal advances over the last 50 years, computers still don't possess many skills and capabilities that most humans take for granted, such as perceiving an image and discern its patterns.

	Because humans possess complex conceptual intelligence, perceptual capabilities and pattern recognition that modern algorithms and machine learning can not yet mimic, computers can not be used to solve these problems for us; human interaction is needed. If we could use a large number of ordinary humans as biological processors in a distributed computer-like system, we could solve real world large scale problems that would otherwise be considered impossible or too time consuming.

	Seeing how popular video games are, what if we designed a game whose purpose is outsourcing computationally difficult functions to humans in an entertaining way? These \emph{games with a purpose} are an implementation of citizen science and ways for regular people outside the scientific community to contribute to scientific research in various fields, and entertaining themselves in the meantime.

\subsection{Human Protein Atlas}

	This project aims to finish analyzing and classifying the images in one subpart of the The Human Protein Atlas (HPA), the Subcellular Atlas \cite{subcellatlas}.

	The Subcellular Atlas contains around five hundred thousand, high resolution, multicolor images of immunofluorescently stained cells that reveal spatial expression patterns at the subcellular level. Around half of those images have already been classified by scientists at the HPA the past few years, and by way of citizen science, we can classify the rest of these images much faster, freeing up the time of these scientists.

	The HPA is a scientific program based in Sweden, and is funded by the Knut and Wallenberg foundation in order to allow for a systematic exploration of the human proteome using antibody-based proteomics. The HPA contains information for a large majority of all human protein-coding genes regarding the expression and localization of the corresponding proteins based on both RNA and protein data.

	The HPA consists of four subparts: normal tissue, cancer, subcellular and cell lines with each subpart containing images and data based on antibody-based proteomics and transcriptomics.

	If this project succeeds, the HPA can provide a better database of images to scientists around the world, since the database is open to researchers anywhere to use. This can have drastic improvements in the fields of biological research, and really helps in understanding how diseases such as cancer operate, which in turn helps in the way of finding a cure. 

	Moreover, if this project succeeds with an acceptable accuracy, the same method can be used on other parts of the HPA, in fact, this whole project has been designed with that in mind, to ease the transition into classifying different kinds of data.

\subsection{MMOS}

	Massively Multiplayer Online Science (MMOS) \cite{mmos}, the name of the concept, and the name of the Swiss company behind it, challenges the way that citizen science is carried out today. 

	While current citizen science projects either create entire new games or "gamify" menial tasks for each research project, the MMOS method is to inject research projects into existing, popular games as seamless parts of their gaming experience (including mechanics, narrative, and visuals). The MMOS method benefits from the time that players already spend on existing games. Even if only a small fraction of that time can be directed to scientific projects, that can provide a huge network of human volunteers to scientific research.

	For this project, MMOS provided an API, which served as a thin interface between the scientific research data from the Human Protein Atlas and the MMORPG EVE Online. It took care of everything related to the citizen science problems, such as task allocation to players, tracking and scoring player performance, giving feedback to the Project Discovery reward system and aggregating results for scientific research. This interface made it easier for us to integrate Project Discovery into EVE Online, since we could focus on coding the client itself into EVE Online. This also substiantially lowered the entry barrier for CCP to implement this feature.

	The MMOS method addresses the core challenges of citizen science games for the following reasons:

	\begin{itemize}
	  \item {\bf Engagement:} There is a large user base already engaged, motivated to solve complex tasks.
	  \item {\bf Motivation and retention:} The motivation of helping science is vital to citizen science projects. However, to keep up the long-term engagement, integrating it with in-game reward systems and other game mechanics is vital and adds an additional layer of motivation.
	  \item {\bf Separation of concerns:} Researchers can do their research work (defining tasks, analyzing results, communicating in research related issues) and professional game designers handle creating the in-game experiences.
	  \item {\bf Reliability:} Popular, massively multi-player games tend to last for many years, with some running for more than a decade with hundreds of thousands of devoted players.
	\end{itemize}

	This approach can offer a win-win situation that benefits all parties: scientists, gamers and the gaming industry at large.

\subsection{CCP Games and EVE Online}

	CCP Games \cite{CCP} is one of the biggest video game developers and publishers in Iceland. CCP is best known for producing virtual worlds such as the multiple award winning game EVE Online, a player-driven science fiction Massively Multiplayer Online Roleplaying Game (MMORPG) \cite{mmorpg}, set in space. EVE Online is home to over 7,000 star systems and is frequently played by hundreds of thousands of players, with tens of thousands playing every day. CCP showed much enthusiasm for this project and was willing to see if the EVE Online community would show positive reactions to helping real world science while playing their favorite game. If a prototype would be well received, CCP was willing to host this game fully on the EVE Online live servers where the whole EVE universe would be able to participate in the project, activating hundreds of thousands of brilliant human brains to solving the scientific problem at the Human Protein Atlas.

	With this project, CCP would be the first MMORPG to incorporate a real world scientific project within itself, pioneering a change that could shake the foundations of modern game development, setting an example to other large games around the world to dedicate their energy to helping the scientific community of the world. Truly, if every major game development company would do the same as CCP has done, these large scientific problems would cease to exist.

	CCP Games graciously allowed us access to two work stations within their headquarters in Reykjavík, and placed us in an EVE development team with senior EVE developers who were available to us when we needed help with the project. They allowed us full access to the headquarters, along with the cafeteria with complementary lunch, which allowed us to stay in house and focus on the development of Project Discovery.

	EVE Online was an excellent fit for this project, since EVE is a science fiction game set in space, a scientific project like this would not feel out of context to players. Especially since the subcellular atlas from the Human Protein Atlas was chosen especially because its images complement the theme of EVE Onlinse and look truly stunning within the client, as shown in figure \ref{fig:zoom}. Furthermore, EVE Online players are already used to getting quite technical and difficult in-game tasks, such as the hacking mini-game. EVE Online's gameplay is also well suited to mini-games of this sort since there are periods in the game which players spend waiting and players could utilize those waiting periods to analyze a few images.

	In order to entice EVE players to play our \emph{game with a purpose}, we offer EVE Online currency for each task solved. That way the game offers the players something that is already of value to them, something that other citizen science games were unable to do, since the rewards are usually only based within the game itself.

\subsection{Previous work}
	This project was a continuation of work done by Reykjavík University graduates Gunnar Þór Stefánsson and Þór Adam Rúnarsson. They worked on it as their final project for the spring semester of 2015 in collaboration with the same companies as we did for this project.

	They created a web prototype very quickly which allowed for basic functionality with a connection to the MMOS API. This prototype was showcased at EVE Fanfest 2015 \cite{fanfest} and was very well received by EVE Online players and CCP alike, which is why they decided to start integrating the prototype into the EVE Online client immediately. They used their remaining time for the semester to focus on integrating the client into EVE Online, and managed to create a crude version of the game before the semesters end.

	They continued their work on the client during the summer of 2015 with a research grant from Rannís, The Icelandic Center for Research. In the beginning of July, Gunnar Þór Stefánsson left the project, and Hjalti Leifsson replaced him. The work continued until Þór Adam Rúnarsson left the project and Jóhann Örn Bjarkason joined in the end of August, when Reykjavík University offered to continue the project as a Final Project.

	When we started the project, the game itself was at a prototype stage within the EVE Online client. The user interface had a hexagonal category selection as well as header and footer elements according to the current design. The game was able to connect to the MMOS API using HMAC-SHA256 authentication, receive new tasks and submit classifications. A simple rewarding system was already in place, rewarding players with in-game currency (ISK), experience points (XP) and loyalty points (LP) when players submitted a solution. When the client received a response from the API, it calculated the appropriate rewards, and displayed a rewarding screen. Finally, a rudimentary tooltip based tutorial phase was implemented, which explained the interface to the user. Nothing else was implemented at that point in time, and that is where we started work on the project, implementing the remaining features according to design with help from CCP Games and MMOS.