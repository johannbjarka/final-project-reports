\section{Introduction}\label{sec:introduction}

Many scientific projects produce large amount of data that can be beneficial for other research projects to analyze, tag or classify. Most often, this work can be conducted by a computer program, given that the structure of the data is relatively simple. In other cases, time- and space complex algorithms need to be applied which may not yet exist nor be executable on computers as we know them today. Many of these classification problems can be solved by the human eye, but for large data sets that can be a time consuming and expensive task.

The amount of data to analyze can be tremendous and it could take scientists years to work through them. In most cases, the analysis of such data is simple enough for humans with minimal training to perform. If ordinary citizens could be motivated to solve real world scientific problems like analyzing data, we could release the huge bottleneck that is throttling the modern scientific community of the world.

That's where Games with a purpose (GWAPs) come in. In GWAPs people perform basic tasks which cannot be automated, they might not be personally interested in solving an instance of a computational problem, instead they simply wish to be entertained~\cite{GWAP}.

The purpose of this project was to design and develop a GWAP based game to help a project named the Human Protein Atlas\footnote{\url{http://www.proteinatlas.org/}}, to improve their publicly available database of images of human tissue and cells, by having ordinary citizens play a game for fun and meanwhile, analyze and identify protein structures in human cells.

Using games to motivate citizens to do science has already been proven useful in the past, with games such as FoldIt~\cite{foldit}, where players compete to find the lowest energy state of a protein, thus helping research in the prediction of protein structures, and Galaxy Zoo~\cite{galaxyzoo}, where players help each other classifying astral bodies in order to find out how galaxies are formed.

These games were successful and people all over the world enjoyed them while competing against each other though scoreboards. However, a problem arose when new players, who were interested in the concept, tried the game out for a short period of time and then never revisited the game. That is, these games had relatively bad retention rate. ~\cite{sauermann}

One possible, and yet to be proven, solution to increase user retention is to inject such games into an already existing game with a large player base, where people are returning daily to play, have been doing so for years, and will hopefully continue to do so for years to come. This way we hoped to solve the difficult problem from previous citizen science~\cite{citizenscience} projects by giving players in-game rewards that they can use outside the project, something that was already of value to them in the existing game. This provides further motivation, a source of in-game rewards for solving relatively simple tasks is something no player should miss out on.

The game that we injected our GWAP based game into was \emph{EVE Online}, the hugely successful massively multiplayer online role-playing game (MMORPG), made by the Icelandic video game developer CCP Games~\cite{CCP} . \emph{EVE Online} is played by tens of thousands of people around the world at any given time~\cite{nature}, so it provides a great platform for a GWAP to reach a tremendous amount of people.

% I think that it would be strong to state here that in this work you are implementing this game within the Eve client. I feel that it would be logical to add that part here! You have described it well what you are doing but you never mentioned that this is an client within the Eve client.

In this report, we seek to illustrate and outline the nature of the project, and the work done by us in the available time frame. In section~\ref{sec:background}, we discuss the problems the project faced and the solutions we offered, as well as listing and discussing the numerous partners to the project, and what benefits they brought. Finally, we explain the previous work already committed to the project before our arrival.

In section~\ref{sec:projectdiscovery}, we explain the gameplay of \emph{Project Discovery}, as well as outlining the network communications necessary for the project, the architecture of the code, and finally we discuss the \emph{Project Discovery} website that we created alongside the game. 

In section~\ref{sec:userevaluation}, we discuss a user test that we performed with an early version of \emph{Project Discovery}. Its purpose was to expose any game breaking bugs and test the user experience of the newly created tutorial interface, alongside the overall gameplay. The results of the user test were positive, we learned that the tool-tips we used contained too much text, and that the tutorial needed to be more difficult. We also noticed that players got less interested in getting the correct solution when they did not receive immediate feedback in some cases. As a result, we changed the tutorial to be harder and shortened the text on every tool-tip, so that new players will be more inclined to actually read them. We also give players immediate feedback for every submission, to keep the player interested for longer.

In section \ref{sec:workscheduleandflow}, we discuss the methodology we used which was Scrum, why we chose it and how it worked for us during the project. We also go over the initial plan for the project, in terms of the team's capacity and the stories we intended to implement. Finally we discuss the actual progress of the project for its duration.

In section~\ref{sec:futurework}, we discuss the future work for this project, and how the \emph{Project Discovery} client has been designed with the express goal in mind to be able to incorporate other science projects into \emph{EVE Online}.

Finally, in section~\ref{sec:conclusion}, we conclude this report, and discuss what we learned from this experience, what we would do better if we could do it again, and how the overall progress went. We also thank the various collaborators that helped us along the way, and showcase the attention that \emph{Project Discovery} has garnered in Iceland and abroad.