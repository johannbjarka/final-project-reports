\section{Introduction}\label{sec:introduction}

Many scientific projects require the analysis of huge amounts of data that the modern computer can not process, since they lack the basic perceptual and cognitive skills that ordinary humans possess from birth. Since human interaction is needed, the scientists working on the project have been analyzing the data themselves. 
% Tölvur geta unnið úr miklu data, þurfum að skýra betur að það er vissur týpur af gögnum
The amount of data needing analysis is so huge that it would take the scientists years to process it all. In most cases, the analysis of such data is simple enough for ordinary humans with minimal training to perform. If ordinary citizens could be motivated to solve real world scientific problems like analyzing data, we could release the huge bottleneck that is throttling the modern scientific community of the world. How can this be done?

This project's purpose was to design and develop a game to help a Swedish scientific program called The Human Protein Atlas, to improve their publicly available database of images of human tissue and cells, by having ordinary citizens play the game for fun and meanwhile, analyze and identify protein structures in human cells.

This concept of using games to motivate citizens to do science has already been proven useful in the past, with games such as \emph{FoldIt} \cite{foldit}, where players compete to find the lowest energy state of a protein, thus helping research in the prediction of protein structures, and \emph{Galaxy Zoo} \cite{galaxyzoo}, where players help each other classifying astral bodies in order to find out how galaxies are formed.

These games got a massive following and went viral for months, with people all over the world competing to get the highest score. However, a problem arose, new players who were interested in the concept, tried the game out, solved a few tasks and then closed the game and rarely ever tried again. How can we keep players interested in playing the game and analyzing the data that the scientific community desperately needs?

In order to try and solve this problem of player retention, we injected our game into an already existing game with a large player base, where people are returning daily to play, have been doing so for years, and will hopefully continue to do so for years to come. This way we hoped to solve the difficult problem from previous citizen science projects by rewarding players with in game currency that they can use outside the project, something that was already of value to them in the existing game. This provides a tangible financial motivation, a reliable source of in game income for solving relatively simple tasks is something no player should miss out on.

% Reference to FoldIt and Galaxy Zoo, state purpose of specific project and discuss continuation of previous work. Vantar roadmap (in this section we do stuff, next section we do other stuff), gefa user eval niðurstöður strax. 