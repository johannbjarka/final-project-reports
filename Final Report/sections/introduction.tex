\section{Introduction}\label{sec:introduction}

Many scientific projects produce large amount of data that can be beneficial for other research projects to analyze, tag or classify. Most often, this work can be conducted by a computer program, given that the structure of the data is relatively simple. In other cases, time- and space complex algorithms need to be applied which may not yet exists nor be executable on computers as we know them today. Many of these classification problems can be solved by the human eye, but for large data sets that can be a time consuming and expensive task.

The amount of data to analyze can be tremendous and it could take scientists years to work through them. In most cases, the analysis of such data is simple enough for ordinary humans with minimal training to perform. If ordinary citizens could be motivated to solve real world scientific problems like analyzing data, we could release the huge bottleneck that is throttling the modern scientific community of the world.

% Add minor introduction to GWAP here with citation. It can be used to link the 
% two paragraphs above together and the one below.

% 
The purpose of this project is to design and develop a game to help a project named The Human Protein Atlas\footnote{\url{http://www.proteinatlas.org/}}, to improve their publicly available database of images of human tissue and cells, by having ordinary citizens play a game for fun and meanwhile, analyze and identify protein structures in human cells.

This concept of using games to motivate citizens to do science has already been proven useful in the past, with games such as FoldIt~\cite{foldit}, where players compete to find the lowest energy state of a protein, thus helping research in the prediction of protein structures, and Galaxy Zoo~\cite{galaxyzoo}, where players help each other classifying astral bodies in order to find out how galaxies are formed.

These games got a massive following and went viral for months, with people all over the world competing to get the highest score. However, a problem arose, new players who were interested in the concept, tried the game out, solved a few tasks and then closed the game and rarely ever tried again. How can we keep players interested in playing the game and analyzing the data that the scientific community desperately needs?

In order to try and solve this problem of player retention, our game has been injected into an already existing game with a large player base, where people are returning daily to play, have been doing so for years, and will hopefully continue to do so for years to come. This way we hoped to solve the difficult problem from previous citizen science projects by rewarding players with in-game currency that they can use outside the project, something that was already of value to them in the existing game. This provides a tangible financial motivation, a reliable source of in-game income for solving relatively simple tasks is something no player should miss out on.

In this report, we seek to illustrate and outline the nature of the project, and the work done by us in the available time frame. In section~\ref{sec:background}, we discuss the problems the project faced and the solutions we offered, as well as listing and discussing the numerous partners to the project, and what benefits they brought. Finally, we explain the previous work already committed to the project before our arrival.

In section~\ref{sec:projectdiscovery}, we explain the gameplay of Project Discovery, as well as outlining the network communications necessary for the project, and the architecture of the code for the project, and finally we discuss the Project Discovery website that we created alongside the game. 

In section~\ref{sec:userevaluation}, we discuss a user test that we performed with an early version of Project Discovery, it's purpose was to expose any game breaking bugs and test the user experience of the newly created tutorial interface, alongside the overall gameplay. The results of the user test were positive, we learned that the tool-tips we used contained too much text, and that the tutorial needed to be harder, as well as players got less interested in getting the correct solution when they did not receive immediate feedback in some cases. As a result, we changed the tutorial to be harder and shortened the text on every tool-tip, so that new players will be more inclined to actually read them, as well as giving players immediate feedback for every submission, to keep the player interested for longer.

In section~\ref{sec:workscheduleandflow}, we outline the scheduled work hours for the project, which split into two parts, the twelve week term and the three week term. We discuss the methodology we used, Scrum, why we chose it and how it worked for us for this particular project. Finally we discuss the actual progress of the project during the given time frame.

In section~\ref{sec:futurework}, we discuss the future work for this project, and how the Project Discovery client has been designed with the express goal in mind to be able to incorporate other science projects into EVE Online, to ease the transition of research projects, in the case of EVE Online players actually solving the problem in its entirety.

Finally, in section~\ref{sec:conclusion}, we conclude this paper, and discuss what we learned from this experience, what we would do better if we could do it again, and how the overall progress went, thank the various collaborators that helped us along the way, and showcase the attention that the Project Discovery project has garnered along the way.
