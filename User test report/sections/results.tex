\section{Results}\label{sec:results}
In this section we analyze the data received, derive from that how Project Discovery could be improved, and what we plan to do about it.

\subsection{Tooltips}
Proper use of tooltips is paramount to the players understanding, players seemed to be confused about their role in the game. We need tooltips to be clear and helpful to players understanding the interface. To this end, tooltips should be utilized not only at the first opening of the interface, but also throughout the whole tutorial \& training session, gradually teaching players about the game as they face the actual problems. The tooltips also need to be shorter and should always focus on one specific message. Testers did not immediately grasp that they could change the colour filter on the main image, we need to add a tooltip that describes this functionality. All in all the tooltips were not descriptive enough, and testers felt that there were not enough of them.

\subsection{Tutorial}
Testers felt that the tutorial was too easy and did not teach them very much about the real game. To rectify this, we are planning to implement a more extensive tutorial, where each image teaches the player a specific attribute of the game. Whether that lesson is about the interface, or how to classify certain patterns in images. 

Testers expressed the need for an explanation after each tutorial task, so we will contact the Human Protein Atlas about getting explanations for each training task, to explain why the correct solution was correct.

\subsection{Interface}
Testers did not have much to say about the interface except that they would have liked to have a zoom option on the main image, and to rectify this, we are going to implement a magnification feature on the main image.

We observed testers, who had completed the training phase, and were contributing to unknown tasks, did not get a result screen. That is normal behaviour since the solution is unknown, we can not reliably give a result. We saw that when users weren't getting instant feedback they were not as interested and seemed less motivated to find the right solution to the tasks. Our solution to this problem was that since we do know the community consensus for specific tasks at any given time, we can show exact percentages for choices by other players. We felt that since testers expressed excitement in seeing a result screen, we should implement this for unknown tasks, so that the game would not lose the factor of excitement once players reach the stage of solving images that have no known solution. This could be a major improvement in the enjoyment players experience while playing the game.