\section{Results}\label{sec:results}
In this section we analyze the data received, derive from that how Project Discovery could be improved, and what we plan to do about it.

\subsection{Tooltips}
Proper use of tooltips is paramount to the players understanding, players seemed to be confused about their role in the game. We need tooltips to be clear and helpful of the interface.

To this end, tooltips should be utilized not only at the first opening of the interface, but also throughout the whole tutorial \& training session, gradually teaching players about the game as they face the actual problems.
The tooltips need to be shorter and should focus always on one specific message.

Testers did not immediately grasp that they could change colours on the main image, we need to add a tooltip that describes this functionality.

All in all the tooltips were not descriptive enough, and testers felt that there were not enough of them.

\subsection{Tutorial}
Testers felt that the tutorial was too easy, and did not teach them very much about the real game.

To rectify this, we are planning to implement a more extensive tutorial, where each image teaches the player a specific attribute of the game. Whether that lesson is about the interface, or how to classify certain attributes of images. 

Testers expressed the need of an explanation after each tutorial task, so we will contact the Human Protein Atlas about getting explanations for each training task, to explain why the correct solution was correct.

\subsection{Interface}
Testers did not have much to say about the interface except that they would very much liked to have a zoom option on the main image, and that they would have liked a result screen for unknown tasks. 

To rectify this, we are going to implement a magnification glass zoom feature, and a result screen for unknown tasks that shows the community consensus so far on that task, in percentages. 