\section{Tools and facilities}\label{tools}

\subsection{Facilities}
All of the work on the project took place at CCP headquarters at Grandagarður 8. Both team members were assigned work stations close to each other for the duration of the project. The work stations included all the tools needed to complete the project. 

\subsection{Tools for the game itself}
We used Perforce for Version control. The reason for that was simple, it's the version control system used at CCP. The programming language we used was Python, again because that's what CCP use to code EVE Online. The integrated development environment (IDE) we used was PyCharm, it integrates very well with Perforce and is widely used at CCP. For code reviews we used Code Collaborator which is also used at CCP.

\subsection{Tools for the website}
We used Perforce for version control for the website as well as the game. The website was done in HTML5 as we wanted it to have a modern look and functionality. For all the scripting purposes of the site we used Javascript as it has universal browser support and we were quite experienced with it. We used the React Javascript library to build the user interface. It limits rendering of objects as it only renders objects that change and it allowed us to have reusable components which simplified the process. To help us keep the Javascript code quality in check we used JSHint. For styling the website we used Less, which allowed us to have shorter and cleaner code. The IDE we chose to use for all the aforementioned languages was Sublime Text, we both had experience using it and felt comfortable with it. 

We used Bower to keep track of packages and dependencies for the website. For unit testing we used Jasmine as our test framework and Karma as our test runner. We used Gulp as our task runner to run various tasks such as JSHint checking, running unit tests, minifying and concatenating the code, each time we changed the code.

\subsection{Other tools}

All of the reports related to the project were done using \LaTeX. We used Toggl to keep track of the total time spent on the project and on individual tasks. To keep track of our Scrum progress, we used Jira, which is used within CCP for that purpose.