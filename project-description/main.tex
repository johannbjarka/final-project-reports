\documentclass{article}
\usepackage[T1]{fontenc}
\usepackage[utf8]{inputenc}
\usepackage{amsfonts}
\usepackage{amssymb}
\usepackage{amsmath}
\usepackage{graphicx}
\usepackage{url}

\author{Hjalti Leifsson (hjaltil13@ru.is)\\Jóhann Örn Bjarkason (johannob01@ru.is)}
\title{\textbf{Project Discovery}}
\begin{document}
\maketitle
\section{Introduction}\label{sec:introduction}

Many scientific projects have gathered huge amounts of data that require the analysis of humans, since modern computers still lack the basic perceptual and cognitive skills that ordinary humans possess from birth, and besides great progress in machine learning recent years, these complex algorithms can not yet produce accurate analysis of this scientific data. Since human interaction is needed, the scientists working on the projects have been analyzing the data themselves.

The amount of data needing analysis is so huge that it would take the scientists years to process it all. In most cases, the analysis of such data is simple enough for ordinary humans with minimal training to perform. If ordinary citizens could be motivated to solve real world scientific problems like analyzing data, we could release the huge bottleneck that is throttling the modern scientific community of the world. How can this be done?

This project's purpose was to design and develop a game to help a Swedish scientific program called The Human Protein Atlas, to improve their publicly available database of images of human tissue and cells, by having ordinary citizens play the game for fun and meanwhile, analyze and identify protein structures in human cells.

This concept of using games to motivate citizens to do science has already been proven useful in the past, with games such as \emph{FoldIt} \cite{foldit}, where players compete to find the lowest energy state of a protein, thus helping research in the prediction of protein structures, and \emph{Galaxy Zoo} \cite{galaxyzoo}, where players help each other classifying astral bodies in order to find out how galaxies are formed.

These games got a massive following and went viral for months, with people all over the world competing to get the highest score. However, a problem arose, new players who were interested in the concept, tried the game out, solved a few tasks and then closed the game and rarely ever tried again. How can we keep players interested in playing the game and analyzing the data that the scientific community desperately needs?

In order to try and solve this problem of player retention, our game has been injected into an already existing game with a large player base, where people are returning daily to play, have been doing so for years, and will hopefully continue to do so for years to come. This way we hoped to solve the difficult problem from previous citizen science projects by rewarding players with in-game currency that they can use outside the project, something that was already of value to them in the existing game. This provides a tangible financial motivation, a reliable source of in-game income for solving relatively simple tasks is something no player should miss out on.

We are two students from Reykjavík University, and this project was our final project for the fall semester of 2015. The work we did for this project was a continuation of what two previous students had done as their final project for the spring semester of 2015, where they developed a web prototype and after showcasing it at EVE Fanfest 2015, started integrating it into the EVE Online client. 

In this paper, we seek to illustrate and outline the nature of the project, and the work done by us in the available timeframe. In section \ref{sec:background}, we discuss the problems the project faced and the solutions we offered, as well as listing and discussing the numerous partners to the project, and what benefits they brought. Finally, we explain the previous work already committed to the project before our arrival.

In section \ref{sec:projectdiscovery}, we explain the gameplay of Project Discovery, as well as outlining the network communications necessary for the project, and the architecture of the code for the project, and finally we discuss the Project Discovery website that we created alongside the game. 

In section \ref{sec:userevaluation}, we discuss a user test that we performed with an early version of Project Discovery, it's purpose was to expose any gamebreaking bugs and test the user experience of the newly created tutorial interface, alongside the overall gameplay. The results of the user test were positive, we learned that the tooltips we used contained too much text, and that the tutorial needed to be harder, as well as players got less interested in getting the correct solution when they did not receive immediate feedback in some cases. As a result, we changed the tutorial to be harder and shortened the text on every tooltip, so that new players will be more inclined to actually read them, as well as giving players immediate feedback for every submission, to keep the player interested for longer.

In section \ref{sec:workscheduleandflow}, we outline the scheduled work hours for the project, which split into two parts, the twelve week term and the three week term. We discuss the methodology we used, Scrum, why we chose it and how it worked for us for this particular project. Finally we discuss the actual progress of the project during the given timeframe.

In section \ref{sec:futurework}, we discuss the future work for this project, and how the Project Discovery client has been designed with the express goal in mind to be able to incorporate other science projects into EVE Online, to ease the transition of research projects, in the case of EVE Online players actually solving the problem in its entirety.

Finally, in section \ref{sec:conclusion}, we conclude this paper, and discuss what we learned from this experience, what we would do better if we could do it again, and how the overall pogress went, thank the various collaborators that helped us along the way, and showcase the attention that the Project Discovery project has garnered along the way.
\section{CCP}\label{sec:ccp}

\section{Game with a purpose}\label{sec:gwap}

\section{Project Discovery}\label{sec:project_discovery}
In this section we discuss the status of the game as it was at the beginning of the project, and outline the architecture of the game  MMOS API.
\subsection{Beginning status of the game}
When the project began, the newest user interface design had been decided, and the game had around half of that design already implemented. Some features had already been implemented, such as players being able to receive images, selecting their appropiate categories and submitting their classifications. A simple rewarding system was already in place, rewarding players with in game currency (ISK), experience points (XP) and loyalty points (LP). A rudamentary tutorial phase was already in place, but needed reforming.\\

\begin{figure}[p]
	\centering
	\graphicspath{ {./graphics/} }
    \centerline{\includegraphics[width=15cm]{PD.png}}
    \caption{\label{fig:PD}This was the design of the UI at the beginning}
\end{figure}
\section{Expected outcome}\label{sec:expected_outcome}

Two outcomes were expected by the end of this project.

\begin{enumerate}
	\item{The release of the finished game client, as it was in its second design iteration, on the live \emph{EVE Online} testing server \emph{Singularity}. This involved implementing most of the main features and design, as the game was barely in a playable state when the project started.}
	\item{The release of a website where users can access general information about \emph{Project Discovery}, view a page where they could learn more about how to classify images correctly, and view statistics on how many images the \emph{EVE} community has classified. }
\end{enumerate}

Before we could start working towards those expectations we needed to work on fixing existing flaws within the game client that had come up just before the project began. Flaws such as authentication issues to the MMOS API, images not being uploaded to the Amazon clouds and optimization of the in-game client. We started fixing those errors right away so feature implementation could start as soon as possible.



%%%%%%%%%%%%%%%%%%%%%%%%%%%%%%%%%%%%%%%%%%%%%%%%%%%%%%%%%%%%%%%%%%%%%%%%%%%%%%%%%%%%%%%%%%%%%%%%%%%%%%%
%\section*{Description}
%As stated above, this project started in January 2015 and the first few months of the project served as the final project at Reykjavík University by Gunnar Þór Stefánsson and Þór Adam Rúnarsson. They continued their work with a grant from Rannís. Gunnar had to leave the project in the beginning of July and Hjalti replaced him so Hjalti was already well up to speed when this semester started. 
% 
%To explain our part in the project we first go over what has been done already and the bigger picture of the project. The project is a so called Citizen Science project. Citizen Science (also known as crowd-sourced science) is scientific research conducted by many amateur or non-professional scientists. A lot of research must be performed by humans since no computational alternative exists and Citizen Science can help speed up research of that kind. 
%
%MMOS (Massively Multiplayer Online Science), which is one of the companies behind the project, is trying to find new ways to conduct Citizen Science by mobilizing players in online video games to take part in the research. The idea is to make the research a seamless part of the gameplay and offer in-game rewards to encourage players to take part. This project, which is called \emph{Project Discovery} within \emph{EVE Online}, focuses on creating a game with a purpose in the \emph{EVE Universe} which will serve as a platform for research tasks. In this incarnation \emph{Project Discovery} will involve players in identifying protein patterns in images of cells and then classifying them in the correct category. This is done in conjunction with \href{http://www.proteinatlas.org/subcellular}{The Human Protein Atlas} which provides the images to be analyzed. In the future Project Discovery could be used for other similar research tasks within the EVE universe.
%
%\section*{Current status of the project}
%
%When we took over the project in the beginning of the semester a lot of the work on the mini-game had already been done.
%The game has an API from MMOS it can talk to in order to receive images for players, and score them based on their performance (in-game it is called "Accuracy Rating"). Players can therefore receive tasks and make selections on that task, based on available categories. They can then submit the task, and get-in game rewards and an updated accuracy score. Players receive a message, thanking them for their contribution, after each submitted task. They also receive in-game experience. Players level up with experience and can figuratively reach an endless level, there is no set limit. The highest level, for which you receive rewards for achieving, is level 100. Players also receive milestone titles for reaching certain level thresholds, such as: "Novice Analyst".
%
%\begin{figure}[H]
%	\centering
%    \includegraphics[width=15cm]{PD.png}
%    \caption{\label{fig:PD}This is the design of the UI at this point}
%\end{figure}
%
%\section*{Outcome of the project}
%
%Two main outcomes are expected by the end of the project. One is to implement the remaining features of \emph{Project Discovery}, and launch it on the \emph{EVE Online} test server, \emph{Singularity}. The other is to develop and launch a website dedicated to \emph{Project Discovery} that details its progress and player statistics, as well as informs the players and general public about it.\\
%
%These are the main remaining features that we expect to be able to finish for this project:
%\begin{itemize}
%  \item A training phase where players get easy tasks and a few categories to solve.
%  \item A detailed rewarding system, where players complete daily challenges for further rewards.
%  \item A leaderboard, where players compete to be the best of the best.
%  \item A detailed view of the history of player submissions, where each player can see what submission changed their score.
%\end{itemize}
%
%We also need to clear up a few things regarding authorization, optimization, player statistics tracking and how our client will be able to connect to the external MMOS API.\\
%
%After release on the test server (or before, it is undecided), our task is to develop a website to launch along with the mini-game. At this point no development or design has been done on the website and only very basic discussion has taken place on the content of it, so the web development part of the project is likely to change and expand in scope. What we know currently is that it will contain information about \emph{Project Discovery} and it will in some way show the progress the \emph{EVE} players are making with the research, for an example how many pictures have been analyzed and how many more need to be analyzed.
%
%\section*{About the companies (from the project proposal)}
%CCP is a leading independent developer of massively multiplayer games, and has
%been praised for its artistry, game design and unique player-driven, infinitely scalable
%storytelling narratives. In addition to \emph{EVE Online}, CCP also develops \emph{DUST 514\textsuperscript{\textregistered}}, a
%groundbreaking, free-to-play, massively multiplayer online first-person shooter for
%the PlayStation\textsuperscript{\textregistered}3, and \emph{EVE: Valkyrie™}, a multiplayer spaceship dogfighting shooter, both set in the EVE Universe. Founded and headquartered in Reykjavik,
%Iceland, in 1997, CCP is privately held, with additional offices in Atlanta, Newcastle,
%and Shanghai. For more information, visit \href{http://www.ccpgames.com/}{ccpgames.com}.\\
%
%MMOS is a privately held start-up company from Switzerland specializing in the field
%of citizen science. The company was founded in 2014 by Attila Szantner and
%Bernard Revaz. Mr. Szantner has a long history in the field of IT, amongst other
%projects being one of the creators of iwiw.hu (the Hungarian “Facebook” which at its
%height \href{http://en.wikipedia.org/wiki/IWiW}{reached more than 4 million users}). Mr.
%Revaz has 15 years of research history in physics (University of Geneva, University
%of California, EPLF).\\
%
%
%CADIA is a highly innovative interdisciplinary centre at Reykjavík University that
%explores and extends the relationship between humans and intelligent machines
%through deep understanding and modeling of human behaviour, design of real-time
%sensing and decision making mechanisms, and extensive prototyping and evaluation
%methodology. CADIA’s projects have received numerous international awards and
%honours, including being three times winners of the international General Game
%Playing (GGP) competition, two times recipients of the Kurzweil Award for Artificial
%General Intelligence (AGI) and placing fourth at the international Autonomous
%Underwater Vehicle (RoboSub) competition. CADIA contains 8 faculty members and
%over 40 research staff and students. Learn more at \href{http://cadia.ru.is/}{cadia.ru.is}.
\end{document}
